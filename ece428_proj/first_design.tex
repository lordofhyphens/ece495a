\section*{First Design}
The first design considered is a basic IIR filter with a critical delay path of one multipler and one adder. This design includes a simple data truncation at the output Y[N] such that the lower order bits are discarded completely. Refer to Figure \ref{subfig:block1-a} for the general block diagram. operands are eight bits, outputs of the multipliers are sixteen bits, and all adders are sixteen bit. The input and output waveforms, see Figure \ref{subfig:output1-b}, indicate that this design is not optimal, however, we will see that further optimization also had little impact on this initial result.

\subsection*{Truncation Code}
The code used in the first design to truncate the lower order bits simply only sends the higher order eight bits to the output.\\*
\\*
always @ (posedge clk)\\*
begin	\\*
y = d3[15:8];\\*
end\\*

\subsection*{First Design Timing Analysis}

% Table generated by Excel2LaTeX from sheet 'Sheet1'
\begin{table}[bh]
\caption{Xilinx Timing Report}
\begin{tabular}{c|c}
\centering
           & Timing in (ns) \\
\hline
Minimum Period &     18.684 \\

     Slack &     81.391 \\

Requrement &        100 \\

Data Path Delay &     18.538 \\
\end{tabular}  
\label{tab:timing1}
\end{table}

According to Table \ref{tab:timing1}, the maximum clock frequency for this design has been calculated to be approximately: 81 MHz

\begin{figure}[htp]
  \begin{center}
    \subfigure[Functional block diagram]{\label{subfig:block1-a}\includegraphics[scale=0.30]{block_one.png}}
    \subfigure[Input/Output comparison]{\label{subfig:output1-b}\includegraphics[scale=0.30]{plot1.png}} \\*
  \end{center}
  \caption{First Design Results}
  \label{fig:design1_results}
\end{figure}