\fancyfoot[R]{DP}
\section[Wireless Communication]{Communications Link Subsystem}
\subsection{Overview}
For communications between the device and the users computer it was decided 
that a wireless connection would be used.  This is because it would be more 
convenient for the user to be able to connect to the device wireless instead of
 being tied down by the physical connection.  There were three choices for use:
 the WT 32 Bluegiga made by Bluegiga Technologies, the BG203 made by SparkFun, 
and the Roving Networks Bluetooth made by Roving Networks.  The first thing 
that is to be taken into consideration is the cost of each chip. Look at 
Table\ref{tab:bt_prices} to see the difference in price.

\begin{table}[hbp]
\caption{Comparison between different communications modules \cite{web:wt32-price}\cite{web:bg203-price}\cite{web:roving-price}}
\begin{tabular}{l | c c } 
	System Name & Packaging & Price \\\hline
	WT32 Bluegiga Breakout & Dual-Inline & \$89.95/unit \\
	BG203 & & \$54.95 \\
	Roving Networks & & \$59.95
\end{tabular}
\label{tab:bt_prices}
\end{table}

If the decision was to be made based purely on the price of the chip then the 
BG 203 would have been chosen but it was not.  This is because out of all three
 of these chips the BG203 was not a breakout chip.  A breakout chip is a chip 
that will allow the user to program the chip further to better suit the users 
needs.  Without this feature the BG203 would have had limited use.

The choice of the WT32 Bluegiga over the Roving Networks Bluetooth comes from 
the features of the chip.  Now, there are many common features of the two chips
 like how they are both qualified Bluetooth modules and both work on 3.3 volts.
  But what made the WT32 more preferable was the fact that ``A host processor 
can control the functionality with ASCII commands via UART or USB 
interface'' meaning that a program can be made to give the chip commands 
instead of having to do it manually\cite{web:wt32-price}.  This is why that 
even though the WT32 Bluegiga is the most expensive out of all the chips it is 
the chip that is best for this type of project.

\subsection{Command Protocols}
The program that was written for this works by sending the command over the
 usart.  This is done simply by having a set of commands that were taken from 
the IWRAP user's guide. The low-level commands have been exposed for purposes
of experimentation; they are not required in the normal operation of the 
subsystem. The activation of one of the commands is handled by use of a
single-character opcode that is sent over the serial link to the PC. The SCU
forwards the command to the MCU, which decodes the command and relays back a
pointer in the SCU's memory for the desired command. Data transmission is 
halted and the command-mode escape sequence is then sent over the SCU's serial
link with the WT32 Bluetooth stack, followed by the ASCII command to the WT32's firmware. 
 
  The commands that were chosen were CALL, CLOSE, CONNECT, ECHO, KILL, and
 RESET.  CALL was chosen because the Bluetooth must be called so that one can
 connect with it.  CLOSE was selected so that one can close the connection 
with the Bluetooth.  CONNECT was chosen so that one can connect with the chip.
  ECHO was chosen in case someone needed to echo data through the Bluetooth
 connection.  KILL was chosen in case anyone needs to kill the connection to
 the Bluetooth.  Lastly, RESET was chosen so that the software in the Bluetooth
 can be reset. 
