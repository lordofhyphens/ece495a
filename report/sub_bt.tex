\fancyfoot[R]{DP}
\section[Wireless Communication]
\subsection{Overview}
For communications between the device and the users computer it was decided 
that a wireless connection would be used.  This is because it would be more 
convenient for the user to be able to connect to the device wireless instead of
 being tied down by the physical connection.  There were three choices for use:
 the WT 32 Bluegiga made by Bluegiga Technologies, the BG203 made by SparkFun, 
and the Roving Networks Bluetooth made by Roving Networks.  The first thing 
that is to be taken into consideration is the cost of each chip. Look at 
Table\ref{tab:bt_prices} to see the difference in price.

\begin{table}[hbp]
\caption{Comparison between different communications modules \cite{web:wt32-price}\cite{web:bg203-price}\cite{web:roving-price}}
\begin{tabular}{l | c c } 
	System Name & Packaging & Price \\\hline
	WT32 Bluegiga Breakout & Dual-Inline & \$89.95/unit \\
	BG203 & & \$54.95 \\
	Roving Networks & & \$59.95
\end{tabular}
\label{tab:bt_prices}
\end{table}

If the decision was to be made based purely on the price of the chip then the 
BG 203 would have been chosen but it was not.  This is because out of all three
 of these chips the BG203 was not a breakout chip.  A breakout chip is a chip 
that will allow the user to program the chip further to better suit the users 
needs.  Without this feature the BG203 would have had limited use.

The choice of the WT32 Bluegiga over the Roving Networks Bluetooth comes from 
the features of the chip.  Now, there are many common features of the two chips
 like how they are both qualified bluetooth modules and both work on 3.3 volts.
  But what made the WT32 more preferable was the fact that ``A host processor 
can control the functionality with ASCII commands via UART or USB 
interface'' meaning that a program can be made to give the chip commands 
instead of having to do it manually\cite{web:wt32-price}.  This is why that 
even though the WT32 Bluegiga is the most expensive out of all the chips it is 
the chip that is best for this type of project.

The program that was written for this works by sending the command over the USART.  This is done simply by having a set of commands that were taken from the 
IWRAP user's guide.  All the user needs to make sure of is that the WT32 is in data mode and then all the user has to do is type in the command they want to use and then the program will send it to the chip and then it is done.
