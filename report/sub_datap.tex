% Subsystem report for Datapath control system, typeset for the LaTeX processor.

\section[Datapath Control]{Acquisition Unit Datapath Control}
% Description of function or purpose
\subsection{Description}
The datapath control subsystem regulates the flow of data from the input modules\index{input modules}
 and serializes it for output to the host PC software. It provides a multiplexed bus for analog and/or digital interface modules and also provides a control interface for those modules. 
Table \ref{tab:control comparison} lists an overview of the options considered for the datapath control
 subsystem. Cost refers to the per-unit cost of a given solution, Complexity is the amount of up-front 
design work that would need to be done to achieve basic functionality. TTL Logic and FPGAs are, for 
these considerations, similar. Both require much work to achieve some basic functionality but had very 
few limits in the the types of solutions that could be realized. FPGAs have an additional advantage over
 basic logic gates in that they are able to be relatively easily reconfigured or updating, which would 
make prottotyping simpler. However, both solutions low-level implementations were considered to be too 
time-inefficient for this design. A general purpose "bare" microprocessor was also considered, but was 
quickly dismissed for similar reasons as FPGAs - the infrastructure required to get a general 
microprocessor functioning was also considered time-inefficient.
\begin{table}[bp]
\caption[Controllers]{Comparison of different control methods}
\begin{tabular}{l| c c c c c c}
		Control Type & Cost & Complexity & Flexibility & Speed & Time Required\\ \hline
		Basic logic  &     &      &      &     & \\
		  gate chips & Low & High & High & Low & V. High \\ 
		FPGA & Average & High & High & Average & High  \\
		Microprocessor & Average & High & Average & Average & Average \\
		Microcontroller & Low & Low & Average & Average & Low \\
\end{tabular}
\label{tab:control comparison}
\end{table}

Microcontrollers, however, are ideal for this type of application: they provide a flexible base and 
generally come packaged with different interfacing subsystems of their own for a low cost. In addition, 
many microcontrollers have a C compiler ported for their architecture, as well as the ability to 
reprogram them in-system. Table \ref{tab:MCU capabilities} outlines some of the capabilties of 
two microcontrollers that had been under consideration. While the the two devices are very similar, 
the Atmel MCU was chosen for its more flexible layout, multiple integrated interfaces, and the 
availability of a mature port of the GNU C Compiler for the device (and its family). The PIC required
the use of the vendor-supplied compiler and IDE.
\begin{table}[bp]
\caption[Atmel and PIC MCUs]{Comparison of Atmel ATMega8515 and PIC18F1220}
\begin{tabular}{l| c c c c c c}

	       &      & Pin  & Maximum& Ease   &            &         \\
	Device & Cost & Count& Speed  & of Use & Interfaces & Features\\\hline
	ATMega8515 & Low & 40 & 20Mhz & High & SPI, & Timers\\
	           &     &    &       &      &USART & External Memory\\\hline
	PIC18F1220 & Low & 18 & 40Mhz & Average & USART & Timers \\
	           &     &    &       &      & & 10-bit ADC\\
\end{tabular}
\label{tab:MCU capabilities}
\end{table}
 
