%Subsystem report for the PC Software subsystem
\fancyfoot[R]{NH}
\section[PC Software]{PC Software Subsystem}
\subsection{Description}
	The PC Software subsystem is responsible for performing two major tasks of 
the PC Diagnostics Tool. The first task is the acquisition of data from the 
Acquisition Unit / Datapath Control subsystem and subsequent storage of data on 
the end-user's PC. The second task is the provision of a display/control 
interface for an end-user operating device. A user must be able to send the 
necessary signals to the PC Diagnostics Tool hardware to control the output of 
the tool, and he or she must have a method of organizing and displaying 
acquired data.

A block diagram of the subsystem is shown in Figure \ref{fig:pcsoft diagram}


\subsection[Design Decisions]{Design Decisions}
\subsubsection[Software Packages]{Software Packages}
There are two possible approaches to writing software that addresses both of 
these needs. The first is to utilize proprietary software development 
environments such as Matlab or Labview that bundle together data acquisition 
and Graphical User Interface (GUI) functionality in a relatively easy-to-use 
software package. The second is to manually build custom software to implement 
these functions.

The first approach considered was to use a proprietary signal processing 
environment. Perhaps the two most popular software packages in this area are 
Labview and Matlab. Labview is a development environment produced by National 
Instruments. Its base package offers user-friendly GUI development tools, ample 
drivers for data acquisition into the PC and even some signal processing 
functionality \cite{web:labviewbase}. Matlab is a widely-known computing environment for science 
and engineering applications with extensive signal processing functionality, 
GUI-development tools and a data acquisition toolbox for acquiring data into 
the PC [2]. In other words, both packages offer exactly the needed 
functionality for this subsystem.

These features come at a price, however. The Labview base package costs 
\$1249 \cite{web:labviewbase}, while Matlab costs \$500 for a single-user license with an 
additional \$200 cost for the Data Acquisition Toolbox \cite{web:matlab}\cite{web:matlabdaq}. 
Furthermore, the nature of development environments like Labview and Matlab mean that 
developing a solution in either would result in a user interface that is only usable in 
Labview or Matlab. This means that end-users would be required to have working 
copies of whichever environment was used to develop the UI. Since hobbyists are 
the target market for othe PC Diagnostics Tool, it is not reasonable to expect 
that every user will possess a copy of Matlab or Labview, and it is even less 
reasonable to assume that every user would purchase a copy in order to use the 
GUI. Building standalone applications in either environment is possible, but it 
comes at further cost: the Labview Application Builder for Windows, at \$999 \cite{web:labviewbase} 
or the Matlab Compiler, at \$500 \cite{web:matlabcompiler}. 

With these cost considerations in mind, a proprietary solution was ruled out. 
The next options explored were programming languages and software packages that 
were inexpensive or free and could be used to implement the PC software. It was 
first necessary to break down the two tasks of the subsystem to understand what 
was required. A break-down of necessary functionality required to implement each
 task is shown in Table \ref{tab:required functionality}.


\begin{table}[h]
\caption[Software functionality]{PC Software tasks broken down by the necessary 
functionality to perform each.}
\begin{tabular}{l | p{7cm}}
	Task & Necessary functionality \\ \hline
	Data acquisition and storage & Serial port communications library, File I/O \\ \hline
	Display/control interface & Serial port communications library, GUI library, 
	Signal/waveform plotting
\end{tabular}
\label{tab:required functionality}
\end{table}


Of the required functionality, nearly every modern programming language has a 
capacity for serial port communications, file I/O and building GUIs, so the deciding 
factor would lie elsewhere. Some deciding factors were brainstormed, and languages were 
evaluated based on cost, ease of implementation and whether or not the resulting 
software would work on different operating systems (i.e. whether it was �cross-platform�).  Programming languages considered were C++, Java, and Python. 

The language comparison performed is shown in Table \ref{tab:languages features}. 
Due to non-availability of concrete data, performance of all three languages was 
considered to be �good-enough� for the software's purposes. All three languages here are 
widely-used, and considering again the target market of hobbyists, it can be expected 
that the languages are optimized enough for the performance needed. In considering the 
ease of implementation for each language, it was found that all the languages had online 
documentation available as well as a serial port library. The only criteria on which they
differed was the �level� of the language. Programming languages can be classified by what 
level the code required to perform a given task is abstracted away from the low-level 
hardware details of accomplishing the task, with high-level languages being farther 
abstracted from the details \cite{web:suranathesis}. Higher level languages are therefore 
easier to implement code with because the language compile or interpreter handle things 
like memory management and CPU operations.


\begin{table}[h]
\caption[Language features]{Comparison of programming language features.}
\begin{tabular}{l | c | p{6cm} | c}
	Language & Cost & Ease of implementation & Cross-platform? \\ 
	\hline
	C++ & Free & Online documentation, serial port library, medium-level language \cite{web:cpptut}\cite{web:cppserial} & No \\ 
	\hline
	Java & Free & Online documentation, serial port library, medium-level language \cite{web:javaapi}\cite{web:javaserial} & Yes \\
	\hline
	Python & Free & Online documentation, serial port library, high-level language \cite{web:pydoc}\cite{web:pyserial} & Yes \\
\end{tabular}
\label{tab:languages features}
\end{table}


After careful consideration of each language's features, it was decided to use Python as 
the programming language. Chosing C++ has no obvious benefits while having the exclusive 
drawback of not being cross-platform, and while Java has, for the purposes of this subsystem, 
essentially the same features as Python, a Python implementation should be easier because of 
its status as a high-level programming language.

For the signal/waveform plotting, it was decided from the beginning that, given the target 
market of hobbyists, real-time data display would not be necessary. Furthermore, many hobbyists 
using the PC Diagnostics Tool will have some technical background in engineering, and perhaps 
even some experience with computation software like Matlab. It seems reasonable, then, to 
perform the data plotting in a numerical computational environment in order to allow users to 
easily extend whatever functionality is provided by the default PC software. GNU Octave, a free, 
open source, Matlab-like numerical computing software package, provides such an 
environment \cite{web:octave}.

A list of software packages used for PC Software subsystem is shown in Table \ref{tab:software packages}. 
Python and GNU Octave have been previously discussed. PySerial is a cross-platform serial 
communications library for Python, and pywin32 is a Windows 32 API library for Python which is 
required for pySerial to work on Windows. All software packages are free and open source.


\begin{table}[h]
\caption[Software packages]{Software packages utilized by the PC Software subsytem, along with total cost.}
\begin{tabular}{l | p{8cm}}
	Software Package & Description \\ \hline
	Pyton & Programming language used to implement data acquisition and interface software. \\ 
	\hline
	pySerial & Serial port communications library for Python. \\
	\hline
	pywin32 & Windows 32 API library for Python, required for pySerial on Windows \\
	\hline
	\multicolumn{2}{l}{Total cost: \$0} \\
\end{tabular}
\label{tab:software packages}
\end{table}


\subsubsection[Software Design]{Software Design}

